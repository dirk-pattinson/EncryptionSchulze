\documentclass{llncs}
\usepackage{amsmath,amsfonts,wrapfig,graphicx,caption,url}
\usepackage{subcaption}
\usepackage[all]{xy}
\captionsetup{compatibility=false}
\pagestyle{plain}

\newcommand{\Nat}{\mathbb{N}}
\newcommand{\Z}{\mathbb{Z}}
\newcommand{\st}{\mathsf{st}}
\newcommand{\LFP}{\mathsf{LFP}}
\newcommand{\Pow}{\mathsf{Pow}}
\newcommand{\GFP}{\mathsf{GFP}}

\begin{document}

\title{No More Excuses: Automated Synthesis of Practical and
Verifiable Vote-counting Programs for Complex Voting Schemes}

\author{Thomas Haines \inst{1} \and
      Dirk Pattinson\inst{2} \and Mukesh Tiwari \inst{2}}
\institute{Polyas, Denmark \and
          Research School of Computer Science, ANU, Canberra}
\maketitle

\begin{abstract}


\end{abstract}


\section{Introduction}


\section{Verification and Verifiability} \label{sec:verif}


\section{Legal Aspects of Verification and Verifiability}






\section{The Schulze Method} \label{sec:schulze}

\section{Homomorphic Schulze Method} \label{sec:schulze}


\section{Experimental Results}


\section{Discussion and Further Work}



One aspect that we have not considered here is encryption of
ballots to safe-guard voter privacy which can be incorporated using
protocols such as shuffle-sum \cite{Benaloh:2009:SSC} and
homomorphic encryption \cite{Yi:2014:HEA}. The key idea here is to
formalise a given voting scheme based on encrypted ballots, and then
to establish a homomorphic property: the decryption of the result
obtained from encrypted ballots is the same as the result obtained from
the decrypted ballots.  We leave this to further work.

\bibliographystyle{myplain}
\bibliography{all2,delta2}


\end{document}
