\documentclass{llncs}
\usepackage{amsmath,amsfonts,wrapfig,graphicx,caption,url}
\usepackage{subcaption}
\usepackage[all]{xy}
\captionsetup{compatibility=false}
\pagestyle{plain}

\newcommand{\Nat}{\mathbb{N}}
\newcommand{\Z}{\mathbb{Z}}
\newcommand{\st}{\mathsf{st}}
\newcommand{\LFP}{\mathsf{LFP}}
\newcommand{\Pow}{\mathsf{Pow}}
\newcommand{\GFP}{\mathsf{GFP}}

\begin{document}

\title{No More Excuses: Automated Synthesis of Practical and
Verifiable Vote-counting Programs for Complex Voting Schemes}

\author{Thomas Haines \inst{1} \and
      Dirk Pattinson\inst{2} \and Mukesh Tiwari \inst{2}}
\institute{Polyas GmbH, Germany \and
          Research School of Computer Science, ANU, Canberra}
\maketitle

\begin{abstract}

In order to establish the trust in election scheme it should be auditable by any one from public (public verifiability ? ), but at the same what information is revealed to public is very critical. If ballots are published in plain text then it gives very nice way to audit the election, but depending on voting protocol used in election (preferential methods),  at the same time it leads to  possibility of coercion . We propose Homomorphic Schulze method to count the ballots without revealing any meaningful content from ballot, but at the same time we give enough evidence in terms of zero knowledge proof about correctness of each step used in counting process) (Writing is  hard, so by all means feel free to criticize it, change it, or delete it 
\end{abstract}


\section{Introduction}


\section{Verification and Verifiability} \label{sec:verif}


\section{Legal Aspects of Verification and Verifiability}






\section{The Schulze Method} \label{sec:schulze}

\section{Homomorphic Schulze Method} \label{sec:schulze}



\section{Experimental Results}


\section{Discussion and Further Work}



One aspect that we have not considered here is encryption of
ballots to safe-guard voter privacy which can be incorporated using
protocols such as shuffle-sum \cite{Benaloh:2009:SSC} and
homomorphic encryption \cite{Yi:2014:HEA}. The key idea here is to
formalise a given voting scheme based on encrypted ballots, and then
to establish a homomorphic property: the decryption of the result
obtained from encrypted ballots is the same as the result obtained from
the decrypted ballots.  We leave this to further work.

\bibliographystyle{myplain}
\bibliography{all2,delta2}


\end{document}
