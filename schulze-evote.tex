\documentclass{llncs}
\usepackage{amsmath,amsfonts,wrapfig,graphicx,caption,url}
\usepackage{subcaption}
\usepackage[all]{xy}
\captionsetup{compatibility=false}
\pagestyle{plain}

\newcommand{\Nat}{\mathbb{N}}
\newcommand{\Z}{\mathbb{Z}}
\newcommand{\st}{\mathsf{st}}
\newcommand{\LFP}{\mathsf{LFP}}
\newcommand{\Pow}{\mathsf{Pow}}
\newcommand{\GFP}{\mathsf{GFP}}

\title{Homomorphic Schulze Counting}
\begin{document}
\maketitle

\section{Problem.}

Given a data structure that represents encrypted ballots, to compute
the margin function. There is choice in the data structure that we
use to represent ballots.

\section{Solutions}
\subsection{Vanessa et al's protocol}

This is taken from Appendix B of the shuffle sum paper. Given: the
ballot in preference order. Produces: an encrypted binary matrix
that indicates pairwise preferences for candidates.

\begin{itemize}
\item the preference order ballot (with encrypted candidate numbers)
\item tuples (E(0), *, *) and (E(-1), *, *) where * are suitably chose
\item candidates
\item a shuffle of these types
\item a ZKP that the shuffle is genuine
\item the shuffled tuples, where the 2nd and 3rd component are decrypted
\item evidence that the decryption of the 2nd and 3rd component of the
      tuple is actually correct
\item the matrix constructed from the tuples with decrypted 2nd and 3rd
      entry.
\end{itemize}

\subsection{Making the matrix representation the main data
structure}

Given: the matrix of encrypted pairwise preferences (as main data
structure). Produces: evidence that the matrix is in fact an
encoding of preferences.

\begin{itemize}
\item  evidence that the matrix is binary, i.e. just has 0/1 entries
 using a shuffle + ZKP + decryption
\item  a shuffle of the rows + ZKP
\item  the homomorphic sums of the rows + decryption
\item  homomorphic sum of the diagonal, and diagonally symmetric elements
 + decryption.
\end{itemize}

The idea here is that a matrix which 
\begin{itemize}
\item is binary, i.e. just 0 and 1 entries
\item has zeros in the diagonal
\item the sum $a_{ij} + a_{ji} = 1$
\item has one row summing up to $n-1$, another row to $n-2$ and so
forth
\end{itemize}
is necessarily obtained from assigning preferences.

\subsection{Using a permutation Matrix}

Another idea is that one could represent a ballot by a permutation
matrix, but it's not clear how
\begin{itemize}
\item evidence that an encrupted matrix is really a permutation
matrix
\item given that we know it's a permutation matrix, how to compute
the pairwise preferences.
\end{itemize}

\subsection{Using Shuffle Naively}

This may be too naive, but how about the following: given a ballot
in candidate order
\begin{itemize}
\item decrypt, and construct the matrix of pairwise preferences
\item encrypt the parwise preference matrix
\item pick a random permutation $\sigma$ 
\item shuffle the ballot using $\sigma$
\item shuffle the encrypted pairwise preference matrix using
$\sigma$ (both rows and columns)
\item decrypt the shuffled ballots + the shuffled preference matrix
and give evidence for the fact that they are indeed decrypted.
\end{itemize}

Remark from Jacub: can chose the permutation so that the pairwise
preference matrix is in upper diagonal form (ie. all ones above the
diagonal) so no need to publish the decrypted matrix. 

Even easier:
\begin{itemize}
\item publish encrypted pairwise preference matrix
\item pick a permutation that brings it in upper triangular form
(always possible if ballot is genuine)
\item publish encrypted permuted matrix + evidence that it's a
genuine permutation
\item publish evidence that it is in upper triangular form.
\end{itemize}

\subsection{Jakub's 2nd Solution}

Given: candidate order ballot
\begin{enumerate}
\item publish candidate order ballot: $(c_1, e(p_1), \dots (c_n,
e(p_n))$
\item publish candidate order ballot with encrypted preferences
\item shuffle the last ballot using specific permutation $\sigma$ so that encypted preferences are
ascending
\item publish this last ballot with decrupted preferences.
\item []
\item publish the matrix pairwise preference ballot, indexed by
  candidates $c_1, \dots c_n$ (in ascencing order) and encrypted
  preferences.
\end{enumerate}

To verify the latter, need to show that a shuffle by $\sigma^{-1}$
decrypts to an upper triangular matrix.

\subsection{Using Range Proofs}

Given: the encrypted ballot in candidate order.

\begin{itemize}
\item using homomorphic subtraction, construct the (encrypted)
matrix $m_{ij} = \mbox{pref. of cand. $a$ - pref. of candidate
$b$}$.
\item using decryption, construct the encrypted pairwise preference
matrix $(p_{ij})$
\item publish a shuffle of $(m_{ij})$ and $(p_{ij})$ plus evidence of correctness
of both shuffles
\item for every entry $m_{ij}$, use a range proof that witnesses
that $p_{ij}$ is $+1$ if $m_{ij}$ is positive, $0$ if $m_{ij}$ is
$0$ and $-1$, otherwise.
\end{itemize}
\end{document}




\end{document}
